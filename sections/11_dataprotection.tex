\section{Protezione dei Dati}
\label{sec:dataprotection}
Nel 1995 in Europa ogni stato decideva regole sulla privacy per conto proprio creando così difficile implementare regole trasversali per siti web visibili ovunque.
L'European data protection board ha imposto ad ogni nazione di creare un proprio organo di sorveglianza sulla tutela della privacy e nel 2018 la \acrshort{GDPR} ha permesso di uniformare i diritti dell'utente, il concetto di dato personale, le regole sul tracciamento dell'utilizzo dei dati e multe salate per i trasgressori.
Dall'articolo 82 della \acrshort{GDPR} \virgolette{Ogni controllore coinvolto nel processo deve essere responsabile per i danni causati dalla loro elaborazione nel caso in cui infranga questo Regolamento. L'elaboratore deve essere responsabile per i danni causati dall'elaborazione solo quando non ha ottemperato agli obblighi che questo Regolamento prevede relativamente all'elaborazione o nel caso in cui abbia agito contrariamente alle istruzioni della legge o del controllore}.
E ancora \virgolette{Quando più di un controllore o elaboratore o entrambi i controllori ed elaboratori sono coinvolti nella stessa elaborazione e nell'archiviazione, visti i paragrafi 2 e 3, sono responsabili per ogni danno causato dall'elaborazione, ogni controllore o elaboratore deve essere ritenuto responsabile per l'intero danno in modo da assicurare l'effettivo compenso al soggetto dei dati}.
Per \textbf{Dati Personali} con la \acrshort{GDPR} si intendono anche dati genetici, biometrici e di posizione geografica.
I dati però possono essere considerati personali in alcuni contesti e non personali in altri.\\
Nel caso in cui un sito web registri l'indirizzo IP degli utenti per identificare e reagire ad attacchi l'indirizzo IP è considerato un dato personale.\\
Mario Rossi è un dato personale? Non sempre perché è un nome molto comune.\\
L'uomo alto di mezza età che possiede un Labrador, guida una Fiat Punto e vive al numero 15 è considerabile un dato personale perché permette di distinguerlo dalla massa.\\
Dunque anche l'indirizzo può essere considerato in alcuni casi un dato personale perché ad esempio nelle Pagine Gialle identifica una persona.\\
In sostanza gli stessi dati possono essere considerati dati personali in un caso e dati non personali in un altro.
Quando un'informazione è legata ad un individuo allora si è in presenza di un dato personale.

\subsection{Doveri nel trattamento dei dati}
Sia il data controller che il data processor hanno dei doveri per quanto riguarda la raccolta, l'elaborazione e l'archiviazione dei dati:
\begin{enumerate}[noitemsep]
    \item Devono avere una base legale per elaborare i dati
    \item Devono elaborare i dati per motivi dichiarati e specifici
    \item Devono collezionare solo i dati necessari per lo scopo
    \item Devono tenere i dati solo per il tempo necessario
    \item Devono tenere solo dati accurati
    \item Devono tenere i dati al sicuro
    \item Devono permettere ai data subject di esercitare i propri diritti
    \item Devono mantenere un registro delle attività di elaborazione
\end{enumerate}
Sei possibili basi legali per elaborare i dati sono:
\begin{itemize}[noitemsep]
    \item Consenso: i data subject hanno dato il loro permesso all'elaborazione dei dati personali per una o più motivazioni
    \item Contratto: l'elaborazione è necessaria per la prestazione di un contratto in cui il data subject è parte o per rispondere alla richiesta del data subject prima che stipuli il contratto
    \item Obblighi legali: l'elaborazione è necessaria per proteggere l'interesse del data subject o di altre persone fisiche
    \item Interesse vitale: l'elaborazione è necessaria per proteggere l'interesse vitale del data subject o di altre persone fisiche
    \item Interesse pubblico: l'elaborazione è necessaria per la prestazione del servizio di pubblico interesse o nell'esercizio di autorità ufficiali investite dal data controller
    \item Interesse legittimato: l'elaborazione è necessaria per lo scopo di interesse legittimato perseguito dal data controller o da parti terze ad eccezion fatta per quegli interessi che sono sovrascritti da interessi o diritti fondamentali di libertà del data subject che richiedono la protezione per dati personali in particolare quando il data subject è un bambino
\end{itemize}
Per \textbf{consenso} del data subject si intende liberamente concesso (non dovrebbe esserci una precondizione per registrarsi ad un servizio), specifico (deve essere richiesto il consenso per ogni motivo di elaborazione e attività condotta sui dati), informato (spiegato in modo chiaro e conciso) e non indicato ambiguamente (silenzio assenso, checkboxs pre-abilitati o inattività) attraverso una chiara un'azione affermativa.\\
Per \textbf{Trasparenza} si intende che il data controller deve informare gli utenti su come vengono elaborati i loro dati in modo semplice da capire.
Quando i dati sono collezionati il data controller deve fornire una notifica di privacy con i dettagli del caso.\\
Ci sono anche delle limitazione delle finalità.
I dati personali devono essere collezionati per specifiche, esplicite e legittime finalità e nessun'altra elaborazione può essere fatta per filanità incompatibili con quelle dichiarate.
Finalità compatibili sono archiviazione per interesse pubblico, scientifico, statistico o di ricerca storica.\\
La minimizzazione dei dati è il concetto che prevede che i dati siano adeguati, rilevanti e limitati a ciò che è necessario per fornire il servizio ed il data controller deve accertarsi che lo siano.\\
I dati personali devono essere accurati e tenuti aggiornati se necessario.
Vanno prese tutte le misure del caso per eliminare i dati non accurati o rettificarli senza ritardi.\\
I dati personali vanno archiviati in modo da permettere l'identificazione del data subject non oltre il necessario per la fornitura del servizio.
Possono tuttavia essere archiviati più a lungo solo per motivi di interesse pubblico, scientifico, statistico o di ricerca storica.\\
I dati personali devono essere elaborati attraverso modalità che assicurano adeguata sicurezza compresa la protezione da elaborazioni non autorizzate o illegali e la perdita accidentale, il danneggiamento o la distruzione utilizzando appropriate misure tecniche e organizzative.\\
Il data controller deve essere in grado di dimostrare di attenersi agli obblighi della \acrshort{GDPR} e per fare ciò dotarsi di sistemi che permettono l'accountability.
Inoltre deve rispettare tutto ciò detto sopra per esempio adottando un approccio di tipo data protection by design and default e incaricando un data protection officer.\\

\subsection{Diritti del Data Subject}
Il data subject ha un certo numero di diritti:
\begin{enumerate}[noitemsep]
    \item Diritto di essere informato
    \item Diritto di accedere ai propri dati
    \item Diritto di rettificare i propri dati
    \item Diritto di far eliminare i propri dati se elaborati in modo non legalmente conforme
    \item Diritto di obiettare a certe elaborazioni se non basate sul consenso
    \item Diritto di non essere soggetto a decisioni completamente automatizzate
    \item Diritto di richiedere i propri dati in formato consultabile
\end{enumerate}
La \acrshort{GDPR} richiede anche sistemi e metodologie per denunciare eventuali violazioni ai dati personali; tutte le organizzazioni devono denunciare determinati personal data breaches all'autorità si supervisione relativa.
Questo deve essere fatto nelle 72 ore successive all'essere venuti a conoscenza del breach per quando possibile.\\
Se il breach è ad alto rischio devono anche essere informate gli individui a rischio senza alcun ritardo.\\
Deve esserci in ogni caso un sistema robusto per scovare eventuali breach, investigarli e riportarli attraverso procedure dedicate in modo da rendere più facile il processo decisionale.\\
Deve anche essere mantenuto un registro di tutti i personal data breach a prescindere che richiedano la notifica agli enti preposti.\\

La \acrshort{GDPR} prevede due livelli di sanzioni:
\begin{enumerate}[noitemsep]
    \item Infrazioni non gravi: possono risultare in una sanzione massima di \euro 10 milioni o del 2\% dei guadagni mondiali annuali (es: non aver riportato il data breach alla data protection authority, non aver incaricato un data protection officer, aver elaborato i dati illegalmente)
    \item Infrazioni gravi: possono risultare in una sanzione massima di \euro 20 milioni o del 4\% dei guadagni mondiali annuali (es: violazione dei principi della protezione dei dati, violazione dei diritti del data subject, trasferimento di dati personali ad organizzazioni internazionali o a paesi terzi)
\end{enumerate}