\section{Cyber Kill Chain}
\label{sec:cyberkillchain}
La cyber kill chain è la catena di controllo del sistema che permette di capire se si è sotto attacco e di che tipo di attacco.\\
Si divide in fasi:
\begin{itemize}[noitemsep]
    \item Reconnaissance
    \item Weaponization
    \item Delivery
    \item Exploitation
    \item Installation
    \item Command and Control
    \item Actions on Objectives
\end{itemize}


\subsection{Reconnaissance}
Obiettivo: selezionare ed ottenere informazioni sull'individuo vittima (cliente dell'azienda oppure dell'IT...)
Fase passiva: ottenere informazioni senza interagire con la vittima (whois, shodan, google, social media, mantego)
Fase attiva: ottenere informazioni attraverso l'interazione con la vittima (nmap, port scanning, vulnerability scanners)

\subsection{Weaponization}
Obiettivo: trovare o creare l'attacco per sfruttare la vulnerabilità attraverso l'utilizzo di
\begin{itemize}[noitemsep]
    \item Metasploit
    \item Exploit DB
    \item Veil Framework
    \item Social Engineering Toolkit
    \item Cain and Abel
    \item Aircrack
    \item SQL Map
    \item Malware ad hoc
\end{itemize}


\subsection{Delivery}
Obiettivo: scegliere come inviare l'attacco
\begin{itemize}[noitemsep]
    \item Web site: compromettendo siti web molto utilizzati in modo che la visita inneschi il download del file malevolo
    \item Social Media: utilizzando profili fake per attacchi di ingegneria sociale
    \item User Input: avendo accesso a tastiere ecc... della vittima
    \item Email: inviando il file malevolo per email
    \item USB: lasciando pendrive incustodite
\end{itemize}


\subsection{Exploitation}
Obiettivo: sfruttare una vulnerabilità già nota come ad esempio:
\begin{itemize}[noitemsep]
    \item SQL Injection
    \item Buffer overflow su software già presente nella macchina
    \item Malware
    \item Javascript hijacking cioè redirect attraverso js
    \item User exploitation
\end{itemize}


\subsection{Installation}
Obiettivo: mantenere l'accesso nel sistema vittima attraverso l'utilizzo delle seguenti tecniche
\begin{itemize}[noitemsep]
    \item DLL Hijacking cioè cambiare la libreria legittima di un programma con quella malevola che avvia il malware all'avvio del programma legittimo
    \item Meterpreter
    \item Remote Access Trojan
    \item Registry Changes cioè la modifica dello scheduling d'avvio in modo da avviare il malware in automatico all'accensione della macchina.
    \item PowerShell commands
\end{itemize}


\subsection{\acrfull{C2}}
Obiettivo: stabilire un canale di comunicazione tra attaccante e macchina vittima in modo da manipolare la stessa da remoto ad esempio aprendo canali di comunicazione a due vie attraverso \acrshort{HTTP}/\acrshort{HTTPS} o cloud, agendo su DNS e protocolli email e collegandosi direttamente a server di controllo in possesso degli attaccanti oppure ad altre macchine vittima utilizzate per questo scopo


\subsection{Actions on Objectives}
Obiettivo: azioni intraprese per conseguire l'obiettivo dell'attacco tra cui
\begin{itemize}[noitemsep]
    \item Ottenere i dati dell'utente
    \item Ottenere maggiori privilegi
    \item Ricognizione interna
    \item Muoversi all'interno del sistema
    \item Rubare dati
    \item Distruggere il sistema
    \item Sovrascrivere o corrompere dati
    \item Modificare dati di nascosto
\end{itemize}


\subsection{TrickBot}
\label{subsec:trickbot}
È un trojan avanzato che i criminali spargono principalmente attraverso campagne di email phishing in cui viene inviato un allegato malevolo o un link ad un server malevolo che fa scaricare il malware contenente il payload che una volta eseguito si collega al server \acrshort{C2} e scarica TrickBot.\\
La struttura del malware è modulare tale per cui i moduli del malware lo spargono in giro abusando del protocollo \acrfull{SMB}.\\
Gli attaccanti possono usare TrickBot per:
È capace di rubare dati attraverso un server \acrshort{C2}, minare criptovalute e riconoscere altri host nella rete tracciandone i dettagli.
In quest'ultimo caso a TrickBot è aggiunto un file di configurazione dedicato che ne specifica i passaggi da compiere.
\begin{itemize}[noitemsep]
    \item Scaricare altri malware come \textbf{Ryuk}\footnote{https://en.wikipedia.org/wiki/Ryuk\_(ransomware)} e \textbf{Conti}\footnote{https://en.wikipedia.org/wiki/Conti\_(ransomware)} ransomware
    \item Essere utilizzato come un downloader stile Emotet
\end{itemize}
\begin{figure}[!htp]
    \centering
    \includegraphics[width=1\textwidth]{images/trickbotkillchain.png}
    \caption{fonte \url{https://blog.gigamon.com/2019/03/02/revisiting-prolific-crimeware-to-improve-network-detection-trickbot/}}
    \label{fig:trickbotkillchain}
\end{figure}