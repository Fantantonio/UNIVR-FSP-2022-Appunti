\section{Social Engineering}
\label{sec:socialengineering}
L'ingegneria sociale, è lo studio del comportamento di una persona al fine di carpire informazioni utili.\\
In un'infrastruttura ben protetta il punto debole è spesso l'uomo; il 22\% degli attacchi di phishing utilizza l'ingegneria sociale.\\
Alcune delle figure che utilizzano in modo criminale l'ingegneria sociale sono:
\begin{itemize}
    \item Hacker: l'obiettivo è far installare alla vittima un software per poi accedere a conti correnti ecc...
    \item Identity Thieve: l'obiettivo è rubare i dati della vittima per impersonarla oppure per venderli nel dark web
    \item Scam Artist: l'obiettivo è convincere la vittima ad inviare soldi al truffatore
\end{itemize}
Un caso fortunato è quello della \textbf{Bangladesh Bank} che ha sventato una truffa da un miliardo di dollari grazie ad un errore di scrittura dei truffatori.\\
Meno fortunato invece il presidente francese \textbf{Macron}.
Attraverso l'ingegneria sociale gli attaccanti sono riusciti ad avere accesso al suo account social e hanno invitato i suoi stessi follower a non votarlo.\\
Parecchi danni li ha fatti anche la \textbf{Facebook Lottery Scam} cioè lotterie truffa che si fa forte del nome Facebook per attirare vittime.
I malintenzionati chiedono soldi alle vittime promettendo un regalo di molto più alto valore per ingannarle.

\subsection{Attack Life Cycle}
Il primo passo del ciclo di vita di un attacco è ottenere informazioni.\\
Un sistema per farlo è quello di spiare la vittima (es: spiare persone che lavorano al pc in treno o in luoghi pubblici), un altro è frugare nella spazzatura alla ricerca di post-it e documenti.\\
Il secondo è stabilire relazioni impersonando qualcuno per poi passare al terzo passo che è quello esplorativo in cui si rubano le informazioni e si inseriscono le backdoor ed infine il quarto passo è l'esecuzione finale con cui si ottiene il risultato desiderato.\\
Dal quarto passo si può tornare al primo nel caso in cui l'obiettivo sia recuperare informazioni per svolgere altri attacchi.\\
I tipi di attacchi che sfruttano l'ingegneria sociale sono:
\begin{itemize}[noitemsep]
    \item Phishing: attacco via email in cui è allegato un file malevolo o inserito un link a sito malevolo. Il testo spinge la vittima ad agire in tempo breve e/o sfrutta una finta posizione di forza
    \item Spear phishing: attacco di phishing targhetizzato sul profilo della vittima
    \item Whailing: attacco di phishing mirato a persone di alto profilo (CEO, CFO...)
    \item Viral hoaxe: post o video che stimolano la curiosità dell'utente per reindirizzarlo su un sito malevolo
    \item Virus hoax: alert falsi attestanti la presenza di virus sulla macchina vittima
    \item Vishing: \virgolette{voice-phishing}, attacco di phishing condotto al telefono. A volte viene inviato un messaggio in-app che chiede di richiamare il finto servizio clienti
    \item Impersonation: impersonificazione di autorità, persone più alte in grado o persone di cui ci fidiamo
    \item SMiShing: \virgolette{SMS-phishing}, attacco di phishing condotto via SMS
    \item Tailgating: accedere fisicamente ad un'area riservata senza essere notati. Per esempio rimanendo dietro un vero dipendente con il badge oppure fingendosi un corriere
\end{itemize}

\subsection{Attacchi di Phishing}
Il phishing è il tentativo di acquisire informazioni sensibili come nomi utente, password, dettagli delle carte di credito ed in alcuni casi indirettamente soldi, fingendosi, in una comunicazione elettronica, entità di cui fidarsi.\\
Uno dei siti più copiati per questi motivi è \textbf{Office 365}.\\
Quanto sono effettivi gli attacchi di phishing?
\begin{itemize}[noitemsep]
    \item L'88\% delle organizzazioni nel modo ha riscontrato attacchi di phishing nel 2019
    \item Il 95\% di tutti gli attacchi in reti aziendali sono il risultato di spear phishing di successo
    \item Il 22\% dei data breaches nel 2019 ha coinvolto attività di phishing
    \item Il 97\% degli utenti non può identificare un'email di phishing sofisticata
    \item Il 30\% delle email sono lette dalle vittime
    \item Il 12\% delle vittime ha cliccato il link malevolo o ha scaricato l'allegato malevolo contenuto nell'email
    \item Il 15\% delle vittime sono contattate almeno un'altra volta nel corso dell'anno
\end{itemize}
Un caso molto rilevante nato da attività di phishing è il data breach di \textbf{Target} che ha portato al furto di dati di carte di credito di 41 milioni di utenti.\\
In alcuni casi l'email di phishing simula una condivisione di documenti Google Docs da parte di una persona che conosciamo, in altri è particolarmente evidente che si tratta di una truffa.\\
È importante fare attenzione al reale mittente così come al vero href del link inserito nell'email.\\
\begin{itemize}[noitemsep]
    \item Exploiting Authority: l'attaccante sfrutta il nome di un'autorità per essere più credibile
    \item Exploiting Scarcity: l'attaccante sfrutta un presunto tempo limitato prima che un evento accada (es: scadenza di una promozione o pagamento di una multa)
    \item Exploiting Commitment: l'attaccante sfrutta il fatto che la vittima ottemperi a norme sociali come ad esempio un avviso di comparizione
    \item Exploiting Linking: l'attaccante sfrutta la fiducia nelle relazioni di amicizia della vittima
    \item Exploiting Reciprocation: l'attaccante sfrutta il fatto che la gente tende a compiere un'azione se ottiene qualcosa in cambio
    \item Exploiting Social Proof: l'attaccante sfrutta il concetto che se molte persone hanno già compiuto un'azione allora c'è da fidarsi
\end{itemize}
La prof. ha condotto uno studio assieme ad un ragazzo della magistrale le cui conclusioni sono che:
Il 31.4\% degli impiegati ha cliccato il link nell'email di phishing, il 23.4\% degli impiegati ha inserito le credenziali nel sito di phishing ed il 69.2\% degli impiegati che ha cliccato il link di phishing ha inserito le credenziali nel sito di phishing.
In pratica chi ha cliccato il link ha anche inserito le credenziali.\\
L'exploit che preme sull'urgenza è risultato il migliore (33.8\% vittime di urgenza, 21.5\% vittime di autorità, 15.6\% vittime di phishing senza tecnica di persuasione).\\
Si è notato anche che i link e la firma dell'email non vengono ispezionati con attenzione.
Se si vogliono testare le proprie capacità di riconoscimento delle email di phishing, esiste un servizio dedicato a questo link \url{www.phishingquiz.withgoogle.com}.

I siti di phishing d'altra parte presentano spesso un indirizzo poco legittimo o l'utilizzo di un sotto dominio fraintendibile o una cartella finale contenente la webapp con nome di un dominio legittimo, e mancano di certificato \acrshort{SSL}, quindi di lucchetto \acrshort{HTTPS}.\\
Il sito \url{https://phishtank.org/} permette di sottomettere un link per avere una valutazione indipendente sulla sua reale legittimità.\\
Per bloccare attacchi di phishing un'azienda dovrebbe adottare un approccio a livelli multipli:
\begin{itemize}[noitemsep]
    \item Livello 1: rendere difficile ad un attaccante raggiungere l'utente implementando controlli anti-spoofing in modo che sia più difficile per l'attaccante conoscere gli indirizzi email del personale, tenere conto delle informazioni che vengono rese pubbliche attraverso sito web e social oltre a informare anche i dipendenti che ciò rappresenta una vulnerabilità, filtrare e bloccare email di phishing
    \item Livello 2: Aiutare gli utenti ad identificare e riportare email sospette attraverso la formazione dei dipendenti e creando un sistema che permette ai dipendenti di chiedere aiuto attraverso un sistema di report semplice in una struttura che non li ferisce sentimentalmente in caso di richiesta d'aiuto o d'errore
    \item Livello 3: Proteggere l'organizzazione dall'effetto di una email di phishing non identificata utilizzando la \acrfull{2FA} ed una struttura di permessi gerarchica, proteggendo gli utenti da siti web malevoli, utilizzando un server proxy e un browser web aggiornato oltre che tenendo attivo un antivirus su ogni dispositivo
    \item Livello 4: Rispondere rapidamente agli incidenti definendo un piano di disposta per le differenti tipologie di incidenti includendo le responsabilità legali e rilevandoli velocemente incoraggiando i dipendenti a riportare le attività sospette
\end{itemize}