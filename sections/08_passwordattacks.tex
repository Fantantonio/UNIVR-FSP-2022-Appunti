\section{Attacchi alle Password}
\label{sec:passwordattacks}
La determinazione dell'identità è normalmente basata su una combinazione di qualcosa che la persona conosce (es: password), qualcosa che la persona possiede (es: smart card) o in ciò che la persona è (es: impronte digitali).


\subsection{Autenticazione basata su Token}
L'autenticazione basata su token richiede che l'utente presenti un token per essere autenticato.
Il token può essere di diverse tipologie tra cui:
\begin{itemize}[noitemsep]
    \item Codice a barre
    \item Dispositivi di \acrfull{OTP}: il dispositivo ed il server di riferimento sono sincronizzati temporalmente così che il tempo è utilizzato da entrambi come seed per la generazione della \acrshort{OTP} e per il controllo.
    \item Carte magnetiche strisciabili
    \item Smart Card: i certificati sono salvati nel chip ed il sistema di autenticazione funziona in questo modo:
    \begin{enumerate}[noitemsep]
        \item L'utente inserisce il pin
        \item Il lettore manda una \virgolette{challenge B}
        \item La smart card genera un valore A e firma A e B con la chiave privata
        \item Il lettore verifica la firma con la chiave pubblica
    \end{enumerate}
    Hanno tuttavia lo svantaggio della possibilità di furto e copia dei dati sul chip.
\end{itemize}


\subsection{Autenticazione Biometrica}
La parola biometrico si riferisce a tutte le misure utilizzate per identificare unicamente una persona basandosi su un tratto biologico o fisico.\\
Di norma i sistemi biometrici incorporano un sistema per scansionare o leggere le informazioni biometriche per poi compararle con quelle di persone a cui permettere l'accesso salvate in memoria.\\
I requisiti per i sistemi di autenticazione biometrica devono essere univoci e non cambiare nel tempo.\\
Le impronte digitali autenticano con un certo margine d'errore ma esistono anche altri sistemi biometrici:
\begin{itemize}[noitemsep]
    \item Firma
    \item Impronta digitale
    \item Scansione della retina/iride
    \item DNA
    \item Analisi della firma
    \item Riconoscimento vocale
    \item Riconoscimento facciale
    \item Analisi della camminata
\end{itemize}
È stato anche proposto l'elettrocardiogramma come sistema biometrico ma ci sono varie problematiche tra cui l'errore di autenticazione del 9\%.
L'autenticazione biometrica ha però delle forti limitazioni tra cui l'accuratezza degli algoritmi (falsi positivi che permettono l'accesso a persone non autorizzate e falsi negativi che la vietano a personale legittimo).
Inoltre i tratti si possono replicare come le impronte digitali dato che le lasciamo in giro e c'è sempre una fetta di popolazione che non accetterebbe questo tipo di controllo.


\subsection{Autenticazione basata su Password}
L'autenticazione attraverso password è molto meno costosa ma pur sempre vulnerabile.
È il sistema d'autenticazione più utilizzato poiché basta richiedere il nome utente e la password ed il sistema deve comparare la password con quella presente nel database.\\
Abbinando l'id ad un sistema di permessi è poi possibile garantire l'accesso solo a personale autorizzato.\\
D'altra parte però il 63\% dei data breache è causato da password rubate o molto deboli.\\
La botnet \textbf{Mirai} funziona scansionando continuamente alla ricerca di dispositivi \acrshort{IoT} che sono accessibili da internet, provando ad inserire le credenziali di default dei dispositivi o nomi utente e password standard e se riesce ad entrare li infetta con il malware forzandoli a fare riferimento ad un server \acrshort{C2} che li rende dei bot utili per attacchi \acrshort{DDoS}.\\
Il problema è che gli utenti hanno molti account online e spesso utilizzano la stessa password per l'accesso inoltre molti utenti utilizzano password deboli come combinazioni numeriche semplici, caratteri in sequenza nella tastiera e pattern prevedibili.


\subsection{Attacchi alle Password}
Ci sono quattro tipologie di attacchi per rubare le password:
\begin{itemize}[noitemsep]
    \item Attacchi offline: accedendo al file o al database delle password
    \item Attacchi online attivi: accedendo da remoto al file o database delle password (exploit, malware, SQLInjection)
    \item Attacchi online passivi: intercettando il traffico contenente la password (key logger, \acrfull{MITM})
    \item Attacchi non tecnici: utilizzando l'ingegneria sociale
\end{itemize}
Le password possono quindi essere \virgolette{crackate} in vari modi tra cui il sistema a forza bruta utilizzato principalmente offline, spiando la vittima mentre la inserisce, cercando le stesse dall'interno, utilizzando un key logger, indovinandola, rubando documenti fisici in cui è scritta, convincendo le vittime a rivelarla, intercettandole attraverso la rete.\\
Le password nei sistemi operativi sono salvate in file specifici:
\begin{itemize}[noitemsep]
    \item Windows:
    \begin{itemize}[noitemsep]
        \item Macchine locali: SAM database
        \item \code{C:\textbackslash Windows\textbackslash System32\textbackslash config}
        \item Montato come HKLM/SAM
    \end{itemize}
    \item Linux:
    \begin{itemize}[noitemsep]
        \item Macchine locali: \code{etc/shadow}
    \end{itemize}
    \item Apple:
    \begin{itemize}[noitemsep]
        \item \code{/var/db/dslocal/nods/default/users}
        \item \code{\<user\>.plist}
    \end{itemize}
\end{itemize}
e sono salvate con uno user di riferimento, un SID ed un hash.
Riguardo gli hash ci sono varie tipologie di hashing tra cui LM che salva password fino a 14 caratteri, nel caso i caratteri della password siano meno vengono inseriti caratteri vuoti, poi la password è divisa in due set da 7 caratteri, criptata e ricomposta e NTLM.\\
Gli \textbf{attacchi di forza bruta} possono essere eseguiti su ricerca esaustiva tentando ogni possibile combinazione di simboli fino ad una preimpostata lunghezza oppure assumendo una lunghezza della password probabile e combinando tutte le lettere maiuscole, minuscole, i numeri e i simboli comuni per un totale di 96 caratteri che per una lunghezza di 8 caratteri della password diventano $96^8=7.2*10^{24}$ cioè 7.2 quadrilioni di possibilità.\\
Prima di tentare un attacco del genere conviene tentare con un \textbf{attacco a dizionario} cioè un attacco di forza bruta che tenta tutte le combinazioni presenti in un file \virgolette{dizionario} appunto il quale contiene password comuni e combinazioni probabili.\\
Gli \textbf{attacchi ibridi} uniscono le tipicità del brute force standard con quelle dell'attacco a dizionario prendendo le parole del dizionario e tentando variazioni delle stesse con caratteri speciali e numeri (es: p4ssw0rd, c!@o).\\
In ogni caso questi attacchi sono lenti e richiedono molto tempo per ottenere una risposta positiva o negativa che sia.
Un modo per rendere più veloce il sistema è utilizzando le \textbf{Rainbow Tables} cioè tabelle con gli hash di molte password già computati cosi da controllare se quello della password vittima è presente ed ottenere il corrispettivo non computato.
Lo svantaggio è che occupano molta memoria.\\
Il più famoso strumento per il cracking delle password è \textbf{John the Ripper}.
Supporta molte tecnologie di cifratura comuni su sistemi UNIX e Windows e prevede tre modi di operare: single crack, wordlist e incremental.\\
Per stimare quanto tempo occorre per crackare una password esistono siti web dedicati in cui inserendo la password in chiaro otteniamo il tempo stimato per crackarla.\\
Sistemi diversi di valutazione delle password possono ritornare valutazioni opposte perché alcuni sistemi valutano positivamente password che hanno un tot numero di caratteri, lettere maiuscole, minuscole, numeri e caratteri speciali, altri riconoscono la vulnerabilità di impostazioni del genere che portano l'utente a seguire uno standard (maiuscola all'inizio, numero alla fine...) quindi preferiscono password lunghe.

\subsection{Possibili Contromisure}
Le contromisure per rendere più difficile scoprire le proprie password sono l'aggiungere un numero random, restringere l'accesso ai file delle password solo ad utenti con privilegi e tenere le password cifrate in hash divise dagli ID degli utenti.\\
Un'attaccante intelligente che prende di mira una persona ricerca possibili affetti, hobby ecc in modo da utilizzare quei nomi per comporre ed indovinare le password dopodiché tenta di crackarla attraverso un dizionario e con password popolari.\\
Le password policies possono essere completamente controproducenti perché spingono l'utente e utilizzare dei pattern che rendono più facile l'individuazione della password.\\
Contromisure effettive sono:
\begin{itemize}[noitemsep]
    \item Bloccaggio dell'account dopo un tot di tentativi di accesso (es: 10)
    \item Tempo fisso prima di riprovare l'inserimento successivamente ad un tentativo errato
    \item Controllo di attività di login inusuali con l'opzione di blocco dell'account
    \item Controllo della password scelta. Se già presente nei database di password crackate allora costringi l'utente a scegliere una nuova password
    \item Utilizzo di \acrshort{2FA} con \acrshort{OTP} o sistemi biometrici
\end{itemize}
Per evitare l'intercettazione utilizzare certificati di comunicazione criptata per evitare spoofing e \acrshort{MITM}.
Ad esempio nei servizi web va utilizzato il protocollo \acrshort{HTTPS} invece del normale \acrshort{HTTP}.
Infine è importante fare attenzione agli attacchi di ingegneria sociale come il phishing, il shoulder-surfing (spionaggio), il dumpster-diving (ricerca di informazioni nella spazzatura) prendendo le giuste contromisure come la formazione e l'aggiornamento.
