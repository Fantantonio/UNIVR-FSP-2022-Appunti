\section{Attacchi alle Password}
\label{sec:passwordattacks}
La determinazione dell'identità è normalmente basata su una combinazione di qualcosa che la persona conosce (es: password), qualcosa che la persona possiede (es: smart card) o in ciò che la persona è (es: impronte digitali).


\subsection{Autenticazione basata su Token}
L'autenticazione basata su token richiede che l'utente presenti un token per essere autenticato.
Il token può essere di diverse tipologie tra cui:
\begin{itemize}[noitemsep]
    \item Codice a barre
    \item Dispositivi di \acrfull{OTP}: il dispositivo ed il server di riferimento sono sincronizzati temporalmente così che il tempo è utilizzato da entrambi come seed per la generazione della \acrshort{OTP} e per il controllo.
    \item Carte magnetiche strisciabili
    \item Smart Card: i certificati sono salvati nel chip ed il sistema di autenticazione funziona in questo modo:
    \begin{enumerate}[noitemsep]
        \item L'utente inserisce il pin
        \item Il lettore manda una \virgolette{challenge B}
        \item La smart card genera un valore A e firma A e B con la chiave privata
        \item Il lettore verifica la firma con la chiave pubblica
    \end{enumerate}
    Hanno tuttavia lo svantaggio della possibilità di furto e copia dei dati sul chip.
\end{itemize}


\subsection{Autenticazione Biometrica}
La parola biometrico si riferisce a tutte le misure utilizzate per identificare unicamente una persona basandosi su un tratto biologico o fisico.\\
Di norma i sistemi biometrici incorporano un sistema per scansionare o leggere le informazioni biometriche per poi compararle con quelle di persone a cui permettere l'accesso salvate in memoria.\\
I requisiti per i sistemi di autenticazione biometrica devono essere univoci e non cambiare nel tempo.\\
Le impronte digitali autenticano con un certo margine d'errore ma esistono anche altri sistemi biometrici:
\begin{itemize}[noitemsep]
    \item Firma
    \item Impronta digitale
    \item Scansione della retina/iride
    \item DNA
    \item Analisi della firma
    \item Riconoscimento vocale
    \item Riconoscimento facciale
    \item Analisi della camminata
\end{itemize}
È stato anche proposto l'elettrocardiogramma come sistema biometrico ma ci sono varie problematiche tra cui l'errore di autenticazione del 9\%.
L'autenticazione biometrica ha però delle forti limitazioni tra cui l'accuratezza degli algoritmi (falsi positivi che permettono l'accesso a persone non autorizzate e falsi negativi che la vietano a personale legittimo).
Inoltre i tratti si possono replicare come le impronte digitali dato che le lasciamo in giro e c'è sempre una fetta di popolazione che non accetterebbe questo tipo di controllo.


\subsection{Autenticazione basata su Password}



\subsection{Attacchi alle Password}



\subsection{Possibili Contromisure}


