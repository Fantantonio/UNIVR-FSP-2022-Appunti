\section{Introduzione}
\label{sec:introduzione}
L'obiettivo della sicurezza informatica è proteggere i dispositivi che ognuno di noi utilizza ed i servizi a cui accediamo oltre a prevenire l'accesso non autorizzato alle informazioni personali che teniamo nei device ed online.\\
Gli \textbf{elementi della sicurezza informatica} sono:
\begin{itemize}[noitemsep]
    \item Confidentiality: criptazione
    \item Integrity: modifiche non autorizzate $->$ controlli hash
    \item Availability: \acrfull{DDoS} $->$ più server
    \item Authenticity: furto password $->$ A2F
    \item Accountability: accesso non autorizzato $->$ log di sistema
    \item Safety
\end{itemize}
I \textbf{concetti chiave della sicurezza}:
\begin{itemize}[noitemsep]
    \item Assets: tutto ciò che ha valore per un'organizzazione sia software che hardware che cloud
    \item Vulnerability: un bug, debolezze o errori che compromettono integrità o fornitura di servizio
    \item Cyber Threat: ogni circostanza che può compromettere le operazioni di un'organizzazione, i suoi asset, individui, altre organizzazioni o la nazione attraverso l'utilizzo di attacchi che si servono delle vulnerabilità
    \item Attack: la realizzazione di un threat specifico che impatta su uno o più degli elementi della sicurezza informatica
    \item Threat Actor: attaccante
    \item Risk: livello potenziale dell'impatto del threat e la probabilità che accada
    \item Controlli di sicurezza: gestione e controlli tecnici prescritti per proteggere il sistema
\end{itemize}

\subsection{Attaccanti - Cybercriminali}
Gli attaccanti sono di norma i \textbf{Cybercriminali} tipicamente interessati al profitto illegale.
La tipologia degli attacchi è:
\begin{itemize}[noitemsep]
    \item Malware
    \item Ransomware
    \item Data breaches
    \item \acrshort{DDoS}
\end{itemize}
ed i vettori di attacco sono altri malware, email e botnet.

\subsection{Attaccanti - Nation State}
In alcuni casi però l'attaccante è un \textbf{Nation State} o più d'uno.
Quando ad attaccare sono degli \virgolette{stati}, il capitale per l'attacco è molto alto e quindi di norma il codice malevolo sfrutta una o più \virgolette{Zero-Day}.\\
I Nation State sono interessati in spionaggio, sabotaggio e sovversione (es: elezioni politiche).
Gli attacchi sono effettuati tramite malware molto sofisticati con tecniche di offuscamento avanzate e colpiscono con attacchi di tipo \acrshort{DDoS}, Data breach e malware.\\
Uno dei casi più famosi di attacco condotto da Nation State è \textbf{Stuxnet}, un worm creato si presume da USA e Israele per rallentare lo sviluppo del programma nucleare Iraniano.\\
Un altro esempio è l'attacco russo per screditare la Clinton ed influenzare le elezioni politiche degli USA.\\
Altri ancora sono SolarWinds \cite{VIRSEC_SolarWInd} (vedi la sottosezione \nameref{sec:cyberthreatlanderscape}) e Kaseya.

\subsection{Attaccanti - Hacktivists}
Gli \textbf{Hacktivists}, attivisti, sono invece attaccanti motivati da visioni politiche, credenze religiose, attivismo, ideologie terroriste o divertimento.
Agiscono sfruttando kit di exploit, email e botnet per attacchi come lo sfregio di siti web, la pubblicazione di informazioni confidenziali e i \acrshort{DDoS}.

\subsection{Attaccanti - Insider Threats}
Infine gli \textbf{Insider Threats} sono ex dipendenti che essendo ancora in possesso di credenziali per l'accesso a risorse di valore, o attaccano intenzionalmente per pubblicare le informazioni sul web, installare bombe logiche o rubare e vendere informazioni, oppure postano accidentalmente contenuti classificati o visitano siti malevoli infettando conseguentemente anche la rete aziendale.



