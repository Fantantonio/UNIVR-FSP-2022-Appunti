\section{Privacy per Design}
\label{sec:privacybydesign}
Ci sono 7 principi fondamentali:
\begin{enumerate}[noitemsep]
    \item Proactive not Reactive: prevenire non rimediare
    \item Privacy come impostazione di default
    \item Privacy incorporata nel design
    \item Funzionalità complete
    \item Sicurezza end-to-end e protezione a vita
    \item Trasparenza
    \item Rispetto per la privacy degli utenti
\end{enumerate}

\subsection{Metodologia LINDDUN}
La metodologia \acrfull{LINDDUN}, basata sulla conoscenza, nasce con lo scopo di aiutare gli ingegneri del software con poche esperienze di privacy ad introdurre il concetto di privacy all'inizio dello sviluppo.\\
La prima parte corrisponde alla creazione del diagramma del flusso di dati e alla descrizione di tutti i dati, poi vengono mappate le minacce ed i rischi al diagramma di flusso dei dati identificandole attraverso l'albero delle minacce, infine vengono gestite le minacce.
\acrshort{LINDDUN} aiuta nella fase di riconoscimento delle minacce dando supporto documentale nel mappare le tabelle e con la tassonomia delle minacce così come nella fase di gestione delle minacce fornendo documentazione sulla tassonomia delle strategie di mitigazione delle minacce e sulla classificazione delle soluzioni per la privacy.\\
Ognuna delle parole che compongono l'acronimo \acrshort{LINDDUN} corrisponde ad una minaccia da considerare e gestire.
La stesura di card che considerano le minacce permette di riassumere il tutto in tabella e successivamente di esportare la tabella come grafico tenendo presente le relazioni di fiducia tra elementi se si assume che si comportino come ci si aspetta.\\
Gli step da compiere sono dunque:
\begin{enumerate}[noitemsep]
    \item Creare un modello del sistema
    \begin{itemize}[noitemsep]
        \item Entità $->$ rettangolo
        \item Processo $->$ ellisse
        \item Archivio di dati $->$ parole tra due linee orizzontali
        \item Flusso di dati $->$ freccia che punta dalla sorgente al ricevente
        \item Confine di fiducia $->$ rettangolo tratteggiato che racchiude vare entità, processi ecc...
    \end{itemize}
    \item Mappare il diagramma di flusso dei dati degli elementi come categorie \acrshort{LINDDUN} $->$ tabella con le \virgolette{X} in caso di minaccia
    \item Ricavare e documentare le minacce ad esempio usando una struttura ad albero
    \item Assegnare una priorità alle minacce riscontrate calcolando un punteggio dato da probabilità ed impatto
    \item Documentare una strategia di mitigazione per tutti i casi di rischio alto o critico utilizzando la tabella comparativa \acrshort{LINDDUN}
    \item Scegliere soluzioni avanzate per la tutela della privacy
\end{enumerate}
