\section{Tipi di Malware}
\label{sec:malwaretypes}
La parola \virgolette{Malware} nasce dall'unione delle parole \textit{Malicious} e \textit{Software} e rappresenta un software o un firmware sviluppato per compiere azioni non autorizzate che causeranno un impatto negativo nella confidenzialità, integrità o accesso al sistema di informazioni.\\
I sistemi possono essere infettati attraverso:
\begin{itemize}[noitemsep]
    \item Accesso diretto al sistema vittima $->$ Dischi infetti, USB ecc...
    \item Ingegneria sociale
    \item Phishing, Spear-phishing, Whale-phishing
    \item Visitando un sito malevolo
\end{itemize}

\subsection{Virus}
Sono malware capaci di replicarsi da soli.\\
Richiedono un'azione umana per essere eseguiti.\\
Potrebbero infastidire l'utente o fare delle piccole modifiche alle macchine infette.\\
I software antivirus sono capaci di scovarli.\\
L'infezione può avvenire attraverso l'apertura di un file (pdf, word) contenente una macro malevola oppure mascherandosi come aggiornamenti di software o sistema.
Sono anche capaci di modificare il loro comportamento a seconda dell'\acrshort{OS}.

\subsection{Worms}
Sono simili ai virus ma non infettano e non richiedono azioni da parte dell'utente.\\
Possono diffondersi in più dispositivi attraverso la rete.\\
Di solito sono più pericolosi di un normale virus e colpiscono server sfruttando le falle di configurazione dello stesso.\\
Il loro obiettivo è avere accesso alla macchina vittima.

\subsection{Trojans}
Sono software a tutti gli effetti ma vengono utilizzati per avere accesso alla macchina infetta per poi scaricare altri virus attraverso la backdoor aperta (es: TrickBot).\\
Possono essere utilizzati anche per rubare informazioni personali, file od anche trasformare la macchina vittima in uno zombie.

\subsection{Rootkits}
Questi malware di solito sono installati direttamente sul kernel e riescono quindi a mascherare le chiamate alle \acrshort{API} dell'\acrshort{OS} fatte da altri malware.\\
Permettono anche di avere accesso root alla macchina e vengono utilizzati spesso per mantenere l'accesso alla macchina vittima.\\
Alcuni non possono essere rimossi quindi l'unità deve essere distrutta.

\subsection{Droppers Downloaders}
Sono malware che contengono il vero e proprio malware quindi eseguibili con il compito di scaricare il malware una volta eseguiti sulla macchina vittima come ad esempio gli allegati malevoli.\\
Tipicamente vengono inviati attraverso malspam sotto forma di file Word o Excel.

\subsection{Key loggers}
In questo caso il malware salva su un file locale tutto ciò che viene scritto dalla tastiera della vittima e poi invia tale file all'attaccante.\\
L'obiettivo è scovare le credenziali ad account di vario genere o i dati delle carte di credito e a volte la comunicazione con l'attaccante viene criptata.\\
Tipicamente viene installato assieme ad altri applicativi malevoli per il furto di credenziali oppure inviati come allegato malevolo per email.\\
Key logger popolari sono \textbf{Refog}, \textbf{Revealer} e \textbf{KidLogger}.

\subsection{Bots}
I bot vengono utilizzati per creare delle macchine zombie a cui inviare comandi.\\
In questo modo nascono le botnet che sono utilizzate per eseguire attacchi \acrshort{DDoS} o per spedire spam.
Le botnet sono comandate da un bot master o più d'uno.\\
Tra le più note botnet ci sono \textbf{Mirai} e \textbf{Satori}.

\subsection{Cripto Miners}
I cripto miner sono malware utilizzati per minare attraverso le macchine infette inviando i profitti al portafoglio dell'attaccante.
La maggior parte sono software per il mining open source modificati.\\
Proliferano grazie a botnet e malspam.

\subsection{Ransomware}
Semplicemente criptano tutti i file nel sistema in cui vengono eseguiti e poi mostrano un messaggio all'utente in cui spiegano come fare per pagare il riscatto ed ottenere la chiave di decriptazione o il software di decriptazione.\\
Tipicamente viene accettato il Bitcoin come forma di pagamento per il fatto che è più conosciuto di altre criptovalute.\\
L'esempio più rilevante è il ransomware \textbf{WannaCry} che controllava la possibilità di condividere attraverso \acrshort{SMB}, faceva leva sull'exploit \textbf{EternalBlue}, installava la backdor \textbf{DoublePulsar} ed il ransomware.\\
I ransomware si dividono in tipologie a seconda del comportamento che hanno e di come agiscono:
\begin{itemize}[noitemsep]
    \item Ransomware: cifrano i file
    \item Lockers: bloccano solo l'interfaccia
    \item Master Boot Record: cifrano o modificano l'\acrshort{MBR}
    \item Wipers: cancellano tutti i dati
\end{itemize}
Si compongono di quattro principali componenti che sono il comportamento simile ai Trojan, la funzionalità di criptare e decriptare i file, il meccanismo di estrazione della chiave ed il modulo in interazione con l'utente.\\
Per raggiungere la vittima vengono infettati siti web, sfruttare vulnerabilità di \acrshort{OS} e software oppure arrivano come allegati malevoli.\\
I tipi di cifratura possono essere a \textbf{chiave pubblica} (la stessa chiave di cifratura viene utilizzata per decifrare) oppure a \textbf{chiave asimmetrica} (si cifra con una chiave e si decifra con l'altra corrispondente).\\
È ovviamente importante che la chiave di cifratura venga eliminata dal sistema vittima in modo da non lasciare traccia.\\
Alcuni ransomware utilizzano la stessa chiave per ogni dispositivo, altri hanno la chiave scritta direttamente nel codice.
In entrambi questi casi è possibile risolvere il problema utilizzando la tecnica di forza bruta fino a trovare la chiave di decriptazione.\\
Molto importanti sono invece i \textbf{Kill Switches} cioè dei sistemi implementati dagli attaccanti per evitare che le loro macchine vengano infettate o permettono di bloccare l'esecuzione del ransomware in modo da evitare che venga infettata la stessa macchina più volte oppure sono errori di codice scoperti dai ricercatori di sicurezza informatica.
In ogni caso sono sistemi che si possono utilizzare per disabilitare il malware.\\
Nel caso di WannaCry, è stato scoperto che il malware faceva richiesta ad un dominio non registrato e se riceveva risposta allora si fermava e non infettava la macchina vittima.
Un ricercatore ha quindi acquistato il dominio abilitando così il kill switch.

\subsubsection{Bad Rabbit}
Uno dei più noti \acrshort{MBR} ransomware è \textbf{Bad Rabbit}, simile a \ref{subsub:notpetya} si finge un aggiornamento di Flash e si sparge attraverso il protocollo \acrfull{SMB}, un protocollo per la condivisione di file in una rete che permette alle applicazioni di una macchina di leggere e scrivere file.\\
Scoperto da ricercatori, il malware tenta di scrivere un file sul disco; se la scrittura fallisce allora il processo di criptazione si ferma.\\
In ogni caso il sistema rimane infetto e potrà contaminare altri dispositivi collegati fisicamente o in rete.

\subsubsection{Hidden Tear}
È un ransomware scritto in c\# creato per motivi di studio ed open source ma viene utilizzato per attacchi reali.\\
Per approfondire andare al seguente link: \url{https://github.com/goliate/hidden-tear}.

\subsection{Prevenzione dai malware}
Per prevenire la ricezione di malware è opportuno adottare delle misure di sicurezza:
\begin{itemize}[noitemsep]
    \item Filtraggio delle email: assieme ad uno filtro anti-spam che in automatico blocca le email malevole e rimuove gli allegati malevoli
    \item Intercettazione dei proxy: in modo da bloccare i siti riconosciuti come malevoli
    \item Internet security gateways: sono in grado di ispezionare il contenuto in certi protocolli di comunicazione e scovare malware conosciuti
    \item Lista di navigazione protetta: all'interno del proprio browser web così da bloccare l'accesso a siti riconosciuti come pullulanti di contenuti malevoli
\end{itemize}
Per rallentare il diffondersi di malware è necessario:
\begin{itemize}[noitemsep]
    \item Utilizzare la \acrshort{2FA}
    \item Tenere aggiornati il \acrshort{OS} ed i software installati
    \item Ridurre i privilegi non necessari
    \item Proteggere l'account degli admin creando un account dedicato per vedere le email ed uno per compiere azioni per cui è necessario avere i permessi d'amministrazione
    \item Fare formazione dei dipendenti
    \item Fare spesso un backup multiplo e sicuro disconnesso dalla rete eseguendo una scansione sugli stessi quando è necessario ripristinarli
    \item Aggiornare regolarmente i prodotti utilizzati per fare il backup
\end{itemize}
Nel caso in cui il malware abbia già infettato l'organizzazione:
\begin{itemize}[noitemsep]
    \item Scollegare immediatamente i dispositivi infetti
    \item Spegnere il Wi-Fi disabilitando ogni altra connessione di rete
    \item Resettare le credenziali d'accesso incluse le password
    \item Formattare con attenzione i dispositivi infetti e reinstallare il \acrshort{OS}
    \item Controllare che il backup non contenga il malware
    \item Collegare il dispositivo ad una rete pulita per scaricare, installare ed aggiornare il \acrshort{OS} e gli altri software
    \item Installare, aggiornare ed eseguire il software antivirus
    \item Ricollegarsi alla propria rete
    \item Monitorare il traffico di rete ed eseguire scansioni antivirus per identificare eventuali rimasugli di infezione
\end{itemize}

